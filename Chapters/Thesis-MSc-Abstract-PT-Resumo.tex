% #############################################################################
% RESUMO em Português
% !TEX root = ../main.tex
% #############################################################################
% use \noindent in firts paragraph
\noindent 
A forma como temos vindo a usar e produzir energia tem contribuindo para as alterações climáticas e para o aquecimento global. 
Hoje mais do que nunca, os países da \ac{UE} tentam encontrar medidas para diminuir o consumo de energia e aumentar a utilização de energias renováveis. Por detrás de iniciativas \textit{top--down}, através de políticas dos governos, a transição para uma economia verde requere o desenvolvimento de soluções inovadoras e capazes de mudar a aitude da sociedade perante a energia. O trabalho aqui apresentado, potencia e combina métodos/tecnologias diferentes -- nomeadamene O \textit{Eco--Feedback}, \textit{Energy Trading} e  \textit{blockchain} -- para desenvolver um sistema que pode ser utilizado pelos cidadãos, especialmente pelos que possuem paineis solares, para reduzir o consumo de energia, otimizar a sua produção e para contribuirem para atingir os objectivos de sustentabilidade da \ac{UE}. 
Concretamente, o principal objetivo deste trabalho é desenvolver uma aplicação para comércio de energia, como produto final para um utilizador, que utiliza o \textit{blockchain} como método de pagamento e simultaneamente fornece \textit{feedback} da energia produzia e consumida através de um sistema de baixo custo e fácil de utilizar.
Este projeto foi implementado como parte do \ac{SMILE}, um projeto de investigação impulsionado pela \ac{UE} H2020, e vai ser testado como um futuro modelo de negócio para uma das soluções desenvolvidas no âmbito do mesmo. 
O estudo teve a duração de um mês e o sistema foi utilizado em 9 casas na Ilha da Madeira, os principais objetivos foram: 1) Estudar a aceitação para comércio de energia entre vizinhos na Ilha da Madeira, de forma a perceber se poderá ser utilizado este modelo de negócio no projeto \ac{SMILE}; 2) Avaliar a perceção dos utilizadores acerca do \textit{blockchain} bem como a sua possivel adoção como um método de pagamento. 
