% #############################################################################
% Abstract Text
% !TEX root = ../main.tex
% #############################################################################
% use \noindent in firts paragraph
\noindent The way we have been using and producing energy so far has been contributing to climate change and global warming. Now more than ever, the \ac{EU} countries need to find measures to cut consumption, increase the penetration of renewables energies, and improve efficiency. Besides top-down initiatives from policy-makers and governments, the transition to a low carbon economy requires the development of solutions that involve and empower people by changing their attitude toward energy. The work presented here, leverages and combine different technologies - namely \ac{EF}, \ac{ET} and blockchain - to develop a system that could be used by citizens, and especially solar \ac{PV} owners, to cut consumption, optimize their production and finally play their part in reaching the \ac{EU} sustainability goals.
Specifically, the main goal of this work is developing an end-user application for energy trading, that uses blockchain as a payment method and contemporarily provides \ac{EF} through a low-cost and easy to use system. This project is implemented as part of \ac{SMILE}, an \ac{EU} H2020 funded research project, and will be pilot tested as a future business model for one of the solutions developed in the scope of SMILE.
The one month pilot test, during which 9 households in Madeira Island used the application, has a twofold goal:
1) Assessing users acceptance of energy trading in Madeira Island in order to understand if it could be a viable future business model for the \ac{SMILE} project;
2) Evaluating users understanding  and attitudes toward adoption of blockchain as a payment method.
