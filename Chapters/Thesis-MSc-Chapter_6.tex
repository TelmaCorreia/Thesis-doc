% #############################################################################
% This is Chapter 6
% !TEX root = ../main.tex
% #############################################################################
% Change the Name of the Chapter i the following line
\fancychapter{Conclusions}
\cleardoublepage
% The following line allows to ref this chapter
\label{chap:conclusion1}


\section{Conclusions}

Final remarks, limitations and future work are provided in this chapter. Since, due to temporal limitations, part of the results from the field study couldn’t be included in the present document, all conclusions drawn here are based on results from the analysis of data collected through Fabric.io. 
	
PowerShare is a complete application, ready to be used. It combines a system for energy trading, that uses blockchain technology as a payment method, with a low-cost Eco-Feedback system. The application has been designed drawing on results from literature review, as well as on the analysis of the most relevant projects on both EFS and Energy trading.
As an EFS, PowerShare provides both real-time and historical data (“Home” and “Historico”) [6], and leverages social comparison (“Ranking”) [5] to foster behaviour change, by challenging users to increase consumption from renewables and compete against their neighbours. As demonstrated by data collected through Fabric.io, both functionalities have been highly appreciated by participants in the field test, confirming results from previous studies.
As a energy trading platform, PowerShare automatically matches demand and offer, and perform the payment through IOTA. The use of IOTA is probably the main contribution of the present work. Designed for the Internet of Things, IOTA is a novel technology and PowerShare, to the author knowledge, is the first project testing such technology for the P2P energy market. Indeed, despite there are already several players in the P2P energy trading market (see §2.2.2), none of them is currently using IOTA system. The big advantage of using IOTA for payment is that it simultaneously solves scalability issues (which characterize several other blockchain technologies) and eliminates transaction fees. The last aspect, in particular, is probably the most important for those who want to develop a P2P energy system, since transactions usually involve micropayments. In addition, IOTA has a very low resource requirements compared to other blockchains. 



\section{System Limitations and Future Work}
