% #############################################################################
% This is Chapter 5
% !TEX root = ../main.tex
% #############################################################################
% Change the Name of the Chapter i the following line
\fancychapter{Conclusions}
\cleardoublepage
% The following line allows to ref this chapter
\label{chap:conclusions}

Final remarks, limitations, and future work are provided in this chapter. Since, due to temporal limitations, part of the results from the field study could not be included in the present document, all conclusions drawn here are based on results from the analysis of data collected through Fabric.io. 
	
	
	
Power Share is a complete application, ready to be used. It combines a system for energy trading, that uses blockchain technology as a payment method, with a low-cost Eco-Feedback system. The application has been designed drawing on results from the literature review, as well as on the analysis of the most relevant projects on both \ac{EFS} and \ac{ET}.
As an \ac{EFS}, Power Share provides both real-time and historical data (“Home” and “Historico”) \cite{Giulio2009}, and leverages social comparison (“Ranking”) \cite{Froehlich2010} to foster behavior change, by challenging users to increase consumption from renewables and compete against their neighbors. As demonstrated by data collected through Fabric.io, both functionalities have been highly appreciated by participants in the field test, confirming results from previous studies.
As an energy trading platform, Power Share automatically matches demand and offer, and perform the payment through IOTA. The use of IOTA is probably the main contribution of the present work. Designed for the Internet of Things, IOTA is a novel technology and Power Share, to the author knowledge, is the first project testing such technology for the P2P energy market. Indeed, despite there are already several players in the P2P energy trading market (see \cref{rps}), none of them is currently using IOTA system. The big advantage of using IOTA for payment is that it simultaneously solves scalability issues (which characterize several other blockchain technologies) and eliminates transaction fees. The last aspect, in particular, is probably the most important for those who want to develop a P2P energy system, since transactions usually involve micropayments. In addition, IOTA has shallow resource requirements compared to other blockchains. 



Power Share differs from most of the existing projects due to the high degree of freedom it gives to users. While much P2P energy trading applications are designed to be as much automated as possible, Power Share gives users the chance to set their criteria for trading energy and provides them with all information needed to make an aware choice (via a detailed feedback about consumption and production). Despite the majority of participants in the field test selected the “automatic setting”, adding the “manual” option allowed us to further test users’ attitude to the system and energy trading in general.
As stated before, the process of matching demand and offer is fully automated. This way, users don’t have, for example, to select the supplier that matches their buying criteria anytime they are in need of energy. Also, transactions are managed by a fair algorithm, which guarantees that all member of the community will equally benefit from the system. 



Like most of the existing projects and startups providing platforms for energy trading, also this project needs to leverage on the traditional energy grid, thus it does not include the development of a physical infrastructure for energy sharing. On this respect, it should be pointed out that the context chosen to test the system presents a great constraint which, currently, prevents the use of such a system. As already mentioned, indeed, to protect the grid from Frequency and Voltage fluctuations, prosumers in Madeira island are not allowed to inject their surplus energy in the grid. For this reason, the energy trading has been simulated. Nevertheless, this limitation also represents the significant potential of a system like Power Share for Madeira Island. Considering a scenario where more prosumers are equipped with \ac{BESS} - a solution that is being developed and tested for the scope of SMILE project - an energy trading system, combined with specialized Battery Management Systems, could be used as a resource for balancing the grid. For example, a prosumer can sell part of the energy available in his battery to the \ac{DSO}, which can then use it to respond to a sudden increase in the energy demand.


Due to the limited duration of the field test (one month), it is not possible to test the long-term effectiveness of such a system. This is something that needs to be investigated to better understand the potential of P2P energy trading. Unfortunately, at the time of this writing, only for one of the existing projects (Brooklyn Microgrid), it is possible to find some prior results about users acceptance of the system. Concerning the other pilot projects and startups, no information is available on this aspect, suggesting the need for further research on the impact of P2P energy trading from a user point of view. In this respect, another limitation of this work is the lack of time for completing the evaluation with the users. The field test in Madeira was also meant to pave the way for the study of users attitude towards P2P energy trading, which is an important aspect that still needs to be investigated. Data collected so far shows that the application has been used to perform energy trading. The fact that all users, also those that selected the “automatic setting” (and consequently, those that could have just “installed and forgotten it”), kept interacting with the app for all duration of the test is already a great result. Nevertheless, given the importance of collecting a feedback from users, the complete evaluation of the system will still be performed after the delivery of this document.
The semi-structured interviews will focus on people understanding and acceptance of blockchain technology, as well as on their perception of such technology for P2P energy trading. From a purely technical point of view, integrating blockchain in a system like Power Share is a significant advantage, since it provides a decentralized payment method without a single point of failure, and it is reliable by itself, thus no third parties are needed. As a developer, the author feels the need to point out a weakness of the IOTA technology, that is to say, its immaturity.  Between the beginning and the end of this study until the end the IOTA technology has been upgraded, and better documentation and support material for developers released. However, due to its immaturity, several issues have arisen while testing the system. Among them, we should mention the unexpected length (concerning time) of the payment process via IOTA, which occasionally has led to the need for “re-attaching” the transaction to effectively complete the payment. This, from a developer perspective, makes the payment system difficult to implement, while for users could be quite annoying. 


\section{Future work} 


Power Share draws on novel technology and therefore there is room for improvements in several aspects. A most significant improvement that could be made regards the matching algorithm described in \cref{psetms}. The matching between the buyer and seller could be made more smartly, by considering the historical consumption and production of each user. This should be done by using machine learning algorithms. Another possible upgrade to the system consists in combining blockchain with \ac{IoT}, i.e. running the smart contract directly on connected smart devices (smart meters). This would further decentralize the process by removing the communication layer between \ac{ETMS} and the sensors.
Last but not least, the possibility to register multiple accounts for the same household, with different roles (only one admin but multiple users), as well as having multiple households under the same account, would be a nice to have featured in the future. Currently, the application allows each user to register only one account per household. Nevertheless, in a real-life scenario, all members of the household should be able, for instance, to access production and consumption data. At the same time, who also owns a summer house equipped with solar panels and a \ac{BESS}, for example, should be able to manage that installation too without the need of registering a second account. 



